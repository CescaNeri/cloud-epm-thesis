\documentclass[12pt,a4paper,openright,twoside]{book}
\usepackage[utf8]{inputenc}

\newcommand{\thesislang}{english}
\usepackage{thesis-style}

% version
\newcommand{\versionmajor}{0}
\newcommand{\versionminor}{1}
\newcommand{\versionpatch}{2}
\newcommand{\version}{\versionmajor.\versionminor.\versionpatch}

\begin{document}
	
\frontmatter

% ! TeX root = thesis-main.tex
\title{Title}
\author{Francesca Neri}
\date{\today}

\newgeometry{margin=0.8in}
\begin{titlepage}
	\begin{center}
		% \vspace*{0.2cm}
		
		\large
		\textbf{ALMA MATER STUDIORUM -- UNIVERSITÀ DI BOLOGNA \\ CESENA CAMPUS}
		\\
		\noindent\hrulefill
		\vspace{0.4cm}
		
		%\Large
		Department of Computer Science and Engineering - DISI \\
            \vspace{0.1cm}
        %\Large
		Second Cycle Degree in Digital Transformation Management \\
            \vspace{0.1cm}
            Class: LM-91
		
		\Large
		\vspace{4cm}
		\textbf{
			% The Role of Enterprise Performance Management \\ 
   %          in Modern Businesses: A Case Study of Oracle \\
   %          Cloud EPM Implementation
            Leveraging Oracle Cloud EPM Business Rules \\
            for Efficient Planning: Insights and \\
            Implementation Strategy
		}
		
		\large
		\vspace{2cm}
		Graduation thesis in \\
		\vspace{0.2cm}
		\textsc{BIG DATA AND CLOUD PLATFORMS}
		
		\vspace{5.5cm}
		\begin{minipage}[t]{0.64\textwidth}
			\begin{flushleft}
				Supervisor \\
				\vspace{0.2cm}
				\textbf{Prof. Matteo Francia}
			\end{flushleft}
		\end{minipage}
		\begin{minipage}[t]{0.34\textwidth}
			\begin{flushright}
				Candidate \\
				\vspace{0.2cm}
				\textbf{Francesca Neri}
			\end{flushright}
		\end{minipage}\\
		
		\vfill
		\noindent\hrulefill
		\vspace{0.3cm}
		\large
		
		I Graduation Session
		\\
		Academic Year: 2022-2023
	\end{center}
\end{titlepage}
\restoregeometry


\begin{abstract}	

This thesis explores the role of Enterprise Performance Management (EPM) solutions in modern businesses, ranging from planning, reporting and forecasting purposes. 
%
The project was carried out using Oracle Cloud EPM, a cloud-based platform developed by Oracle Corporation, that offers an integrated suite of financial and operational planning and analysis tools.

The thesis starts by providing an overview of Enterprise Performance Management and its importance in modern businesses. 
%
It then delves into the benefits of using a cloud platform for EPM, including lower costs, increased flexibility, and easier scalability.

The main focus of the thesis regards the exploration of how financial and operational processes are integrated in a single EPM solution, whilst using data coming from different sources. 
%
The project involved working with a real-world client to identify the company planning needs and
then manipulating dimensions, elements, and hierarchies to meet those needs.
%
The techniques used to create customized planning models for the client will be analyzed, including the use of pre-built templates and the creation of custom calculations and formulas.

In addition to the technical aspects of the project, this thesis also explores the business benefits of using an integrated EPM solution, including better decision-making, improved financial performance, and increased visibility into operational processes.

Overall, the goal of this project is to provide a detailed exploration of the role of EPM solutions in modern businesses, with a focus on the importance of using a cloud platform and integrating financial and operational processes in a single solution. 
%
The project, carried out using Oracle Cloud EPM, provides a practical example of how businesses can leverage EPM solutions to improve their financial and operational planning and analysis capabilities.

\end{abstract}

\begin{dedication}
Optional. Max a few lines.
\end{dedication}

% \begin{acknowledgements}
% Optional. Max 1 page.
% \end{acknowledgements}

%----------------------------------------------------------------------------------------
\tableofcontents   
%\listoffigures     % (optional) comment if empty
%\lstlistoflistings % (optional) comment if empty
%----------------------------------------------------------------------------------------

\mainmatter

%----------------------------------------------------------------------------------------
\chapter{\introductionname}
\label{chap:introduction}
%----------------------------------------------------------------------------------------

% https://www.researchgate.net/search/publication?q=enterprise%20performance%20management
% To ask to chatty: can you write a presentation of oracle cloud epm, explaining its functioning and application in the business context

In today's highly competitive business environment, effective performance management is essential for organizations to achieve their strategic goals and stay ahead of the competition.

Performance management provides valuable data and insights that can help businesses make informed decisions about their workforce. 
%
By analyzing performance data, businesses can identify trends, anticipate issues, and make adjustments to their strategy as needed.

Enterprise Performance Management (EPM) is an integral part of the success of modern enterprises as it helps organizations to gain visibility into their performance and to identify areas of improvement.
%
It is a management approach that integrates multiple business processes and systems to enable organizations to plan, measure, analyze, and optimize their performance.

 Enterprise Performance Management enables businesses to improve strategic decision-making, maximize their productivity and achieve the desired business results while ensuring that expenses are minimized and that the best use of resources is made.
 %
 It assists businesses to effectively forecast and plan for their future financial performance, helping them to ensure compliance and reduce risk. 
 
 The advantages of EPM are numerous and make it a valuable tool for organizations of all sizes.

 The aim of this thesis is to investigate the role of EPM in modern businesses and the benefits and challenges of implementing EPM solutions. 
 %
 Specifically, this thesis will focus on a case study of Oracle Cloud implementation in a business to illustrate how EPM solutions can improve organizational performance. 
 
 Oracle Cloud is a leading cloud-based EPM solution that provides organizations with a comprehensive suite of financial and operational performance management applications.
 %
Oracle Cloud EPM provides a flexible and scalable solution that can meet the needs of organizations of all sizes.
%
It offers various tools and solutions that allow companies to plan, budget, forecast, and report on their financial and operational performance.
%
The implementation of an Oracle Cloud EPM solution involves a series of steps throughout which organizations must work closely with their implementation partner to ensure that the solution is tailored to their specific needs and requirements.

Using a cloud platform to implement an EPM solution can provide numerous benefits for the organization. 
%
One of the primary advantages regards the ability to access critical data and applications from anywhere and at anytime.
%
Additionally, cloud platforms can help reduce costs associated with maintaining on-premises infrastructure and provide scalability and flexibility to meet changing business needs. 
%
Cloud-based EPM solutions also enable organizations to leverage the latest technologies and innovations without having to worry about managing and upgrading hardware and software. 
%
Cloud providers typically offer high levels of security and compliance, which can help organizations protect sensitive data and maintain regulatory compliance. 

%
\paragraph{Thesis Structure.}
%

Accordingly, the reminder of this thesis is structures as follows:
%
\Cref{chap:introduction} will provide an overview of the background and significance of the topic, scope and limitations of the study, and an overview of the methodology.
%
\Cref{chap:background} will review the literature on Enterprise Performance Management, including the definition and concepts of EPM and its evolution in modern businesses. 
%
It also provides an overview on Oracle Cloud and its EPM solutions, together with the benefits and challenges of implementing EPM in businesses. 
%
\Cref{chap:design} will describe the approach and methodology used for the implementation of and EPM solution to a company.
%
\Cref{chap:implementation} will present the case study of Oracle Cloud implementation in a business, including the company background and context, EPM solution selection and implementation process, key features and functionalities of the implemented EPM solution, impact of the EPM solution on the company's performance, and the best practices for EPM implementation. 
%
Finally, \Cref{chap:conclusions} concludes the thesis summarizing the main concepts ans discussing the implication for theory and practice.

\section{Background and significance of the study}

Enterprise Performance Management is a management approach that involves using data and analytics to monitor, measure, and improve an organization's performance. 
%
It integrates various management processes, including financial planning, budgeting, forecasting, risk management, and performance measurement, to help organizations align their goals with their strategies and improve their overall performance.
%
To access and analyze data more quickly and accurately, organizations typically use software applications that automate and streamline this process.

These application often include dashboards, pre-built report templates and other tools that provide real-time insights into organizational performance.

The ultimate goal of EPM is to help organizations to achieve their strategic objectives and optimize their performance, whether that be in terms of financial performance, customer satisfaction, operational efficiency, or other key performance indicators. 
%
By using data-driven insights and decision-making, EPM enables organizations to more effectively manage risks, allocate resources, and make strategic decisions that improve their overall performance and competitiveness in the marketplace.

Enterprise Performance Management operates within a complex and dynamic business environment, characterized by a number of key trends and challenges. 
%
One of the most significant trends in recent years has been the growing adoption of cloud technology for enterprise applications, including EPM. 

Cloud-based EPM solutions offer many advantages over traditional on-premises systems, including greater scalability, flexibility, and accessibility, as well as lower costs and reduced IT complexity. 
%
As a result, more and more organizations are turning to cloud-based EPM solutions to improve their performance management processes.

Another key trend in modern business is the growing importance of data-driven decision making. 
%
With the increasing availability of data and advanced analytical tools, organizations are able to gain deeper insights into their operations and performance, and make more informed decisions. 
%
EPM plays a critical role in this trend by providing real-time insights into organizational performance and enabling managers to make data-driven decisions that align with their strategic goals. 
%
This helps organizations to be more proactive in identifying and responding to performance issues, and to optimize their operations and resources to achieve their desired outcomes.

In addition to these trends, EPM also operates within a business environment that is constantly evolving and adapting to changing market conditions. 
%
The need for businesses to be agile and responsive to these changes is an ongoing challenge, and EPM can help address this by providing a flexible and adaptable framework for performance management. 
%
By monitoring key performance indicators and providing real-time insights into performance, EPM enables businesses to quickly identify and respond to changes in market conditions, and to make strategic decisions that keep them ahead of the competition.

Overall, the context in which EPM operates is characterized by a rapidly evolving business environment, where technology, data, and agility are critical to success. 
%
Enterprise Performance Management plays a critical role in helping organizations to adapt to these trends and challenges, by providing a data-driven and agile approach to performance management that enables them to optimize their operations, resources, and strategic decision making.

\section{Scope and limitations of the study}

The scope of the project includes some areas of focus that range from the definition of Enterprise Performance Management, its benefits, its implementation and future trends.
%
The thesis starts by defining what Enterprise Performance Management is and how it differs from the other management approaches in place.

Then, the project investigates the potential benefits that EPM can bring to modern businesses including enhanced decision-making, better alignment of business objectives with strategies, increased efficiency and improved performance.
%
It then explores the challenges and best practices associated with implementing EPM in businesses including issues related to data collection and analysis, performance measurement, reporting and the use of technologies and specific software.

The project aims at analyzing a case study and explores the future trends and challenges that are likely to impact the role of Enterprise Performance Management in relation with emerging technologies, changing business models, and evolving business needs and expectations.

There are some limitations related to the study that should be taken into account.
%
Such limitations include both extrinsic and intrinsic elements, ranging from the access to large amounts of data, the complexity of the topic investigated and the diversity of applications involved.

More in details, in order to explore the impact of Enterprise Performance Management on businesses, it may be necessary to have access to data related to business performance belonging to different organizations.
%
Moreover, modern businesses vary widely in terms of size, sector, business model and many other factors.
%
As a result, it can be difficult to generalize findings in one type of business to other types of businesses.

\section{Overview of the methodology}

The methodology applied to the implementation of an Enterprise Project Management solution follows some key steps and best practices that depend on the specific requirements of the client company, the size and complexity of the project and the available resources.

The project begins with a planning phase, during which the company's business needs and goals are identified and the scope of the project is defined.
%
This involves identifying the objectives and deliverables of the project, as well as any constraints or assumptions that will impact the implementation.
%
During this phases, it is important to identify the stakeholders by determining who will be impacted by the implementation and what their needs and requirements are.
%
Another key element in the planning phase is the definition of success criteria in order to measure the outcome by using performance indicators.
%
Finally, a clear and detailed project plan is developed, outlining the timeline, the budget, and resources required to complete the implementation.

Then, during the solution design phase, the EPM solution is designed based on the identified business requirements and goals, including the configuration of the environment, customizing features and workflows, and integrating with other systems.

The development and testing phase is the core part of the project.
%
In this phase the EPM solution is built and configured,relying on a process of continuous feedback from the customer.
%
Both individual components and the solution as a whole are tested to ensure that they meet the business requirements and goals.

After that, the client company's personnel which will use the system is trained through training sessions, user manuals, and other resources to help users understand the system and use it effectively.

Finally, the solution is deployed to the client's company environment, migrating the data from the old system, integrating the systems and configuring the settings and permissions.
%
After the EPM solution is deployed, ongoing support and maintenance are provided to endure the system runs smoothly and meets the company's needs.
%
This involves providing technical support, troubleshooting issues, and applying updates.

%----------------------------------------------------------------------------------------
\chapter{Literature Review}
\label{chap:background}
%----------------------------------------------------------------------------------------

Write background here.

This section is likely to contain a lot of citations.

\section{Definition and concepts of Enterprise Performance Management}

\section{The evolution of EPM in modern businesses}

\section{Oracle Cloud and its EPM solutions}

\section{Benefits and challenges of implementing EPM in businesses}

%----------------------------------------------------------------------------------------
\chapter{Methodology}
\label{chap:design}
%----------------------------------------------------------------------------------------

Write design here.

% \begin{figure}
% 	\centering
% 	\includegraphics[width=0.5\linewidth]
% 	\caption{A class diagram created with PlantUML}
% 	\label{fig:classes}
% \end{figure}

You may want to reference images in your thesis.
%
In this case, you are encouraged to make them \emph{floating}, and reference them by means of labels.

\section{Scope and objectives of the project}

\section{Needs assessment and project plan}

%----------------------------------------------------------------------------------------
\chapter{Case Study: Oracle Cloud Implementation}
\label{chap:implementation}
%----------------------------------------------------------------------------------------

Write implementation here.

You may need to reference listings in your thesis.
%
In this case, you are encouraged to make them \emph{floating}, and reference them by means of labels.

\section{Company background and context}

\section{EPM solution selection and implementation process}

\section{Key features and functionalities of the implemented EPM solution}

\section{Impact of the EPM solution on the company's performance}

\section{Best practices for EPM implementation}

%----------------------------------------------------------------------------------------
\chapter{\conclusionsname}
\label{chap:conclusions}
%----------------------------------------------------------------------------------------

\section{Summary of the project}

\section{Discussion and concluding thoughts}


%----------------------------------------------------------------------------------------
% BIBLIOGRAPHY
%----------------------------------------------------------------------------------------

\nocite{*} % uncomment this to show all the reference in the .bib file
\bibliographystyle{plain}
\bibliography{bibliography}


\end{document}
